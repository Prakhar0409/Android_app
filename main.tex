\documentclass[12pt]{article}
\usepackage{graphicx}
\usepackage{hyperref}




\begin{document}
\begin{titlepage}
\title{COP290}
\centering
{\scshape\Large COP290\par}
\vspace{1cm}
{\scshape\Huge User Registration Android App\par}

\vspace{3cm}
	{\Large\itshape Mayank Rajoria (2014CS10233)\par}
\vspace{0.3cm}
	{\Large\itshape Prakhar Gupta (2014CS10290)\par}
\vspace{0.3cm}
	{\Large\itshape Prakhar Agrawal (2014CS10207)\par}

\vfill
\raggedleft
Course  Coordinator\\
Vinay Joseph Ribeiro
% \vfill
\end{titlepage}

\begin{abstract}
The purpose of this assignment was to create a basic android app. The android app would contain a simple form which would accept some data like team name and details of its members. This data would be sent to a server provided to us for registration of the said team. The most important part of the application is to be able to send a POST request to the given server with the data and show the response to the user. The application has been made in order to follow the design practices by Google and made as user friendly as possible.
% \vfill
\end{abstract}


% Mention the detailed description of what the application does.

\section{Introduction}
The android app was made using the open source Android SDK provided by Google Inc. The application first provides user with a form. This for asks for the team name to be registered and the names and entry numbers of the team members to be registered.
\par
The app itself provides an auto suggest menu when the user starts entering their names. The suggestions are from among the names of CS Btech and Dual degree students at IIT Delhi from the entry year 2014. On selection of any name, the entry number of the selected student is automatically added. Once this data is entered and the user wishes to register for the course, the app itself first performs a validation on the data provided. All fields are checked whether they are filled or not, whether the names entered contain only alphabets and whether the format of entry numbers are correct.
\par
Upon validation of the data, the data is sent to the server via a POST request using the Volley library for Android. The app receives the response from the server or the android system informing it of the server response of any other network error. These errors are appropriately handled and corresponding messages are shown to the user. Appropriate sounds are also played to provide audio response to the user.
\par
Finally the user is shown a page on successful registration where they are also given an option to share the app with other people via various media.

\section{User Interface}
\par
The application interface is divided into three screens. The main screen that provides the Registration utility is the one that opens up when you open the application. The action button on the top screen provides an option to jump to the about screen that provides information about the app. The third screen is the success message screen that appears when a team successfully submits its information and passes the validation
\subsection{Registration Screen}

\begin{figure}[!ht]
	\centering
% 	\includegraphics[width=0.5\textwidth]{./UserInterface}
	\caption{Registration Screen}
\end{figure}



\begin{itemize}
\setlength\itemsep{-0.4em}
\item This screen presents the registration form to the user. It has 7 text boxes to enter the data.
\item The first text box accepts the team name.
\item The next text box is for the team member's name. Once the user starts entering the data into this text box, suggestions are displayed for the name.
\item The entry number text box takes the entry number of the corresponding student.
\item Finally there is the "Submit" button to send the data to register for the course.
\end{itemize}

\begin{itemize}
\setlength\itemsep{-0.4em}
\item The team name can be anything apart from an empty string.
\item The name of a student can contain only English alphabet characters and space characters.
\item The entry number is checked for its correct format using the language of Regular Expression.
\item All inputs are first trimmed of trailing white spaces before performing the above checks.
\end{itemize}

\begin{itemize}
\setlength\itemsep{-0.4em}
\item When the user starts entering the data into this text box, suggestions are displayed from among the 2014 CSE entry students in IIT Delhi. Upon selecting a name, their entry number is automatically entered in their entry number text box.
\item On pressing the submit button, all of user's data is validated using the above guidelines. and appropriate message is displayed in the form of a Toast in case of errors found.
\item If all the entered data is correct, the data is submitted to the given URL and its response is displayed to the user.
\item Upon successful registration, the user is taken to the corresponding screen.
\end{itemize}

\subsection{Success Screen}

\begin{figure}[!ht]
	\centering
% 	\includegraphics[width=0.5\textwidth]{./UserInterface}
	\caption{Success Screen}
\end{figure}

\begin{itemize}
\setlength\itemsep{-0.4em}
\item This screen basically displays a success message to the user.
\item The user is also provided with a button to share their success and invite more students to register for the course using this app.
\end{itemize}

\begin{itemize}
\setlength\itemsep{-0.4em}
\item Upon clicking the share button an Intent is created which contains a message and a link to be shared.
\item This intent is passed on to the system to create a chooser of various apps to allow the user to choose where to share the message.
\end{itemize}


\subsection{About Screen}

\begin{figure}[!ht]
	\centering
% 	\includegraphics[width=0.5\textwidth]{./UserInterface}
	\caption{About Screen}
\end{figure}

\begin{itemize}
\setlength\itemsep{-0.4em}
\item This screen basic information relating to the application.
\item This screen introduces user to the course COP290 and seeks to popularize it among the application users.
\item This is a stationary UI without much of a functionality apart from information providing.
\end{itemize}




\section{Implementation Details}

\begin{itemize}
\subsection{Posting Data}
\item The crux of the application is the implementation of the POST request and sending this to the provided server http://agni.cse.iitd.ac.in.
\item This has been done using "Volley" which is a networking library for android applications and has a huge upcoming community support.
\item Volley was included to the project and a String request was generated using the data submitted via the form on the Registration page screen.
\item The response String was then parsed into JSON format and then using the response code appropriate response was generated using the android Toast.
\item Methods for network communication: The onClickListener for the button register. It further calls the method register that posts the data to the server upon validation via the checkData() method.
\end{itemize}


\subsection{Validation}
\begin{itemize}
\item Validating data before posting it to the server is an important part of the application as it does not allow fake/invalid users to register for the course via the application interface.
\item This is implemented via the checkData() function in the same module. It checks for names being only sets of strings separated by white spaces.
\item checkEntryNumber() is another function that handles the validation for the Entry number boxes of the application.
\item We check for entry number via the pattern matching in form of Regular expressions.
\item \textbf{Error Cases Handled:}
\begin{itemize}
		\item In TeamName: All Strings except the empty string are accepted. Handled in the method checkData().
		\item In Names: Characters other than [A-Za-z" "] in the entry number are not allowed in the names. Handled in the method checkData().
        \item In EntryNumbers: Check for all errors of incorrect number of characters or incorrect position of letters or invalid entry year. Handled in the method checkEntryNumber(). All probable error cases are extensively handled using simple regular expression.
	\end{itemize}
\end{itemize}
    
    
   
    
\subsection{Autocomplete \& IntelliSense}
\begin{itemize}
\item On writing first few characters of name, the application starts suggesting people it thinks you may be naming. Since we had access to only data of the department people, suggestions are based on this data only.
\item Similar suggestions appear on typing entry numbers using the same AutocompleteTextView.
\item On completing the name field the application tends to complete the entry number field correctly by automatically determining the person's identity based on the name. This is the IntelliSense feature which is unique to our application.
\item This feature is based on our access to the department data from the CSE website. We store entries of student names vs entry numbers which we later use to predict entry numbers based on names.
\end{itemize}

\subsection{Miscellaneous}
\begin{itemize}
\item Keyboard hiding feature has been added, which hides the keyboard whenever the user clicks outside any of the EditText text box.
\end{itemize}

\section{Project repository and guildlines for Contribution}
\par The code for the project is being maintained in this repository: \url{ https://github.com/Prakhar0409/Android\_app.git}.
Read the contribution guidlines given on the github project page to contribute. Don't forget to document changes. 
\\
\textbf{Voila! you have made it till the end.}

\section{Citations}
\begin{itemize}
\item \url{ developer.android.com/training/index.html}
\item \url{www.javatpoint.com/android-tutorial}
\item \url{developer.android.com/training/volley/index.html}
\sloppy
\item \url{http://stackoverflow.com/questions/33573803/how-to-send-a-post-request-using-volley-with-string-body}
\end{itemize}



\bibliographystyle{abbrv}
\bibliography{references}

\end{document}

